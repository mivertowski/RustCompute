% Section 5: Implementation

\section{Implementation}
\label{sec:implementation}

RingKernel is implemented in Rust with approximately 25,000 lines of code across
multiple crates. This section details key implementation aspects.

\subsection{Crate Architecture}

RingKernel is organized as a Cargo workspace:

\begin{itemize}
    \item \texttt{ringkernel-core}: Core traits, types, and enterprise features (457 tests)
    \item \texttt{ringkernel-derive}: Procedural macros for actor definitions
    \item \texttt{ringkernel-cpu}: CPU backend for testing and fallback
    \item \texttt{ringkernel-cuda}: NVIDIA CUDA backend with cudarc bindings
    \item \texttt{ringkernel-cuda-codegen}: Rust-to-CUDA transpiler (190+ tests)
    \item \texttt{ringkernel-wgpu}: WebGPU cross-platform backend
    \item \texttt{ringkernel-wgpu-codegen}: Rust-to-WGSL transpiler
    \item \texttt{ringkernel}: Facade crate re-exporting all functionality
\end{itemize}

\subsection{Rust-to-CUDA Transpilation}

The transpiler converts Rust DSL to CUDA C, enabling developers to write GPU
kernels in familiar syntax while generating optimized device code.

\subsubsection{Input: Rust Actor Definition}

\begin{lstlisting}[language=Rust, caption={Rust actor definition}]
#[ring_kernel(id = "processor", block_size = 128)]
fn handle(ctx: &RingContext, msg: &Request) -> Response {
    let tid = ctx.global_thread_id();

    // Shared memory reduction
    let partial = shared_reduce(msg.values, ReductionOp::Sum);

    ctx.sync_threads();

    if tid == 0 {
        Response { sum: partial, count: msg.values.len() }
    }
}
\end{lstlisting}

\subsubsection{Output: CUDA Kernel}

The transpiler generates a persistent kernel with message handling:

\begin{lstlisting}[language=CUDA, caption={Generated CUDA kernel (simplified)}]
extern "C" __global__ void processor_kernel(
    ControlBlock* cb,
    RingBuffer* h2k_queue,
    RingBuffer* k2h_queue,
    K2KRouter* k2k_router
) {
    __shared__ uint8_t shared_mem[4096];

    // Initialize HLC
    HlcClock hlc = {0, 0, cb->kernel_id};

    // Persistent message loop
    while (true) {
        // Check termination
        if (cb->state == STATE_TERMINATED) break;

        // Try receive from H2K
        MessageHeader header;
        if (try_dequeue(h2k_queue, &header)) {
            hlc_receive(&hlc, &header);

            switch (header.message_type) {
            case MSG_REQUEST:
                Request* req = (Request*)(h2k_queue->data
                    + header.payload_offset);
                Response resp = handle_request(req);

                // Send response
                MessageHeader resp_hdr;
                resp_hdr.correlation_id = header.correlation_id;
                hlc_send(&hlc, &resp_hdr);
                enqueue(k2h_queue, &resp_hdr, &resp);
                break;

            case MSG_TERMINATE:
                cb->state = STATE_TERMINATED;
                break;
            }
        }

        // Update heartbeat
        if (threadIdx.x == 0) {
            atomicExch(&cb->heartbeat, cb->system_time);
        }
    }
}
\end{lstlisting}

\subsubsection{Transpilation Pipeline}

The transpiler operates in phases:

\begin{enumerate}
    \item \textbf{Parse}: Convert Rust source to AST using \texttt{syn}
    \item \textbf{Analyze}: Extract types, identify GPU intrinsics, validate semantics
    \item \textbf{Transform}: Convert Rust constructs to CUDA equivalents
    \item \textbf{Generate}: Emit CUDA C source code
    \item \textbf{Compile}: Invoke \texttt{nvcc} to produce PTX
\end{enumerate}

\subsubsection{Intrinsic Mapping}

The transpiler maps 120+ GPU intrinsics across categories:

\begin{table}[h]
\centering
\caption{GPU intrinsic mapping examples}
\label{tab:intrinsics}
\begin{tabular}{@{}ll@{}}
\toprule
\textbf{Rust DSL} & \textbf{CUDA Output} \\
\midrule
\texttt{ctx.thread\_id()} & \texttt{threadIdx.x} \\
\texttt{ctx.block\_id()} & \texttt{blockIdx.x} \\
\texttt{ctx.sync\_threads()} & \texttt{\_\_syncthreads()} \\
\texttt{ctx.atomic\_add(\&x, v)} & \texttt{atomicAdd(\&x, v)} \\
\texttt{ctx.warp\_shuffle(v, lane)} & \texttt{\_\_shfl\_sync(0xffffffff, v, lane)} \\
\texttt{pos.north(buf)} & \texttt{buf[(y-1)*width + x]} \\
\bottomrule
\end{tabular}
\end{table}

\subsection{Memory Management}

\subsubsection{Mapped Memory for Zero-Copy}

RingKernel uses CUDA mapped memory for H2K/K2H queues:

\begin{lstlisting}[language=Rust, caption={Mapped memory allocation}]
// Allocate mapped memory visible to both CPU and GPU
let h2k_buffer = device.alloc_mapped::<u8>(QUEUE_SIZE)?;
let k2h_buffer = device.alloc_mapped::<u8>(QUEUE_SIZE)?;

// CPU can write directly, GPU sees changes
h2k_buffer.host_ptr().write(message);
// Memory fence ensures visibility
std::sync::atomic::fence(Ordering::SeqCst);
\end{lstlisting}

\subsubsection{Stratified Memory Pooling}

For analytics workloads, RingKernel provides size-stratified buffer pools:

\begin{lstlisting}[language=Rust, caption={Stratified memory pool}]
pub enum SizeBucket {
    Tiny,   // 256 bytes
    Small,  // 1 KB
    Medium, // 4 KB
    Large,  // 16 KB
    Huge,   // 64 KB
}

impl StratifiedMemoryPool {
    pub fn allocate(&self, size: usize) -> StratifiedBuffer {
        let bucket = SizeBucket::for_size(size);
        self.buckets[bucket].try_pop()
            .unwrap_or_else(|| self.alloc_new(bucket))
    }
}
\end{lstlisting}

This reduces allocation overhead for repeated operations.

\subsection{Cooperative Groups Integration}

For grid-wide synchronization, RingKernel uses CUDA cooperative groups:

\begin{lstlisting}[language=CUDA, caption={Cooperative groups for grid sync}]
#include <cooperative_groups.h>
namespace cg = cooperative_groups;

__global__ void persistent_stencil(ControlBlock* cb, ...) {
    cg::grid_group grid = cg::this_grid();

    while (cb->state == STATE_ACTIVE) {
        // Phase 1: Compute
        compute_stencil(local_tile);

        // Grid-wide barrier
        grid.sync();

        // Phase 2: Exchange halos
        exchange_halos(k2k_router);

        grid.sync();

        // Update step counter
        if (threadIdx.x == 0 && blockIdx.x == 0) {
            atomicAdd(&cb->step_counter, 1);
        }
    }
}
\end{lstlisting}

Cooperative launch requires special invocation:

\begin{lstlisting}[language=Rust, caption={Cooperative kernel launch}]
// Check device supports cooperative launch
let props = device.properties();
assert!(props.cooperative_launch != 0);

// Launch with cooperative API
unsafe {
    cudarc::driver::result::launch_cooperative_kernel(
        func,
        (grid_x, grid_y, grid_z),
        (block_x, block_y, block_z),
        shared_mem_bytes,
        stream,
        kernel_params.as_mut_ptr(),
    )?;
}
\end{lstlisting}

\subsection{K2K Message Routing}

Kernel-to-kernel messaging uses a routing table in device memory:

\begin{lstlisting}[language=CUDA, caption={K2K routing}]
struct K2KRouteEntry {
    uint32_t dest_kernel_id;
    uint32_t buffer_offset;
    uint32_t buffer_size;
    uint32_t flags;
};

__device__ bool k2k_send(K2KRouter* router, uint32_t dest,
                          MessageHeader* msg, void* payload) {
    // Find route
    K2KRouteEntry* route = find_route(router, dest);
    if (!route) return false;

    // Enqueue to destination's buffer
    RingBuffer* dest_buf = (RingBuffer*)(router->base
        + route->buffer_offset);
    return enqueue(dest_buf, msg, payload);
}
\end{lstlisting}

For 3D stencil computations, K2K enables halo exchange between neighboring tiles
without host involvement, critical for persistent FDTD simulations.

\subsection{Enterprise Features}

RingKernel includes production-ready infrastructure:

\subsubsection{Health Monitoring}

\begin{lstlisting}[language=Rust, caption={Kernel watchdog}]
pub struct KernelWatchdog {
    timeout: Duration,
    last_heartbeat: Instant,
}

impl KernelWatchdog {
    pub fn check(&mut self, cb: &ControlBlock) -> HealthStatus {
        let heartbeat = cb.heartbeat.load(Ordering::Acquire);
        if self.last_heartbeat.elapsed() > self.timeout
           && heartbeat == self.last_seen_heartbeat {
            HealthStatus::Stalled
        } else {
            self.last_seen_heartbeat = heartbeat;
            HealthStatus::Healthy
        }
    }
}
\end{lstlisting}

\subsubsection{Circuit Breaker}

\begin{lstlisting}[language=Rust, caption={Circuit breaker pattern}]
pub struct CircuitBreaker {
    state: AtomicU8,  // Closed, Open, HalfOpen
    failure_count: AtomicU32,
    threshold: u32,
    reset_timeout: Duration,
}

impl CircuitBreaker {
    pub fn execute<F, R>(&self, f: F) -> Result<R>
    where F: FnOnce() -> Result<R> {
        match self.state.load(Ordering::Acquire) {
            OPEN => Err(Error::CircuitOpen),
            HALF_OPEN | CLOSED => {
                match f() {
                    Ok(r) => { self.record_success(); Ok(r) }
                    Err(e) => { self.record_failure(); Err(e) }
                }
            }
        }
    }
}
\end{lstlisting}

\subsubsection{Observability}

RingKernel integrates with Prometheus and OpenTelemetry:

\begin{lstlisting}[language=Rust, caption={Metrics export}]
// Prometheus metrics
let metrics = PrometheusExporter::new();
metrics.register_counter("ringkernel_messages_processed");
metrics.register_histogram("ringkernel_message_latency_us");

// OpenTelemetry tracing
let tracer = OtlpExporter::new("http://jaeger:4317");
let span = tracer.start_span("process_message");
span.set_attribute("kernel_id", kernel_id);
\end{lstlisting}

\subsection{WebGPU Backend}

For cross-platform support, RingKernel includes a WebGPU backend via wgpu:

\begin{lstlisting}[language=Rust, caption={WebGPU runtime}]
pub struct WgpuRuntime {
    device: wgpu::Device,
    queue: wgpu::Queue,
    compute_pipeline: wgpu::ComputePipeline,
}

impl RingKernelRuntime for WgpuRuntime {
    async fn launch(&self, kernel: &str, opts: LaunchOptions)
        -> Result<KernelHandle> {
        // WebGPU doesn't support true persistent kernels
        // Emulate with host-driven dispatch loop
        let handle = EmulatedPersistentHandle::new(
            self.device.clone(),
            self.compute_pipeline.clone(),
        );
        Ok(KernelHandle::Emulated(handle))
    }
}
\end{lstlisting}

WebGPU limitations (no persistent kernels, no 64-bit atomics) are documented
and worked around where possible.

\subsection{Testing Infrastructure}

RingKernel has 900+ tests across the workspace:

\begin{itemize}
    \item \textbf{Unit tests}: Core logic, transpiler passes
    \item \textbf{Integration tests}: End-to-end kernel execution
    \item \textbf{Property tests}: Queue invariants via \texttt{proptest}
    \item \textbf{GPU tests}: Require hardware, use \texttt{\#[ignore]}
\end{itemize}

\begin{lstlisting}[language=Rust, caption={Property-based queue testing}]
proptest! {
    #[test]
    fn queue_fifo_order(messages: Vec<TestMessage>) {
        let queue = MessageQueue::new(1024);
        for msg in &messages {
            queue.enqueue(msg.clone()).unwrap();
        }
        for expected in &messages {
            let actual = queue.dequeue().unwrap();
            prop_assert_eq!(actual, *expected);
        }
    }
}
\end{lstlisting}
