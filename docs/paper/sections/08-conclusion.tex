% Section 8: Conclusion

\section{Conclusion}
\label{sec:conclusion}

We presented RingKernel, a GPU-native persistent actor model that brings actor
semantics to GPU computing. By treating GPU thread blocks as actors with lock-free
message passing and Hybrid Logical Clocks for causal ordering, RingKernel enables
a new class of interactive GPU applications.

\subsection{Summary of Contributions}

\begin{enumerate}
    \item \textbf{GPU Actor Model Formalization}: We extended the actor model with
    H2K, K2H, and K2K communication channels that map naturally to GPU memory
    hierarchies, providing a formal foundation for GPU-native actors.

    \item \textbf{ControlBlock Architecture}: We introduced a 128-byte GPU-resident
    structure for actor lifecycle management, enabling supervision patterns
    analogous to Erlang's ``let it crash'' philosophy.

    \item \textbf{HLC on GPU}: We implemented Hybrid Logical Clocks for causal
    ordering across thousands of concurrent GPU actors, a novel capability
    enabling distributed systems semantics on parallel hardware.

    \item \textbf{Rust-to-CUDA Transpilation}: We developed a transpiler that
    converts high-level Rust actor definitions to efficient CUDA kernels,
    lowering the barrier to GPU actor programming.

    \item \textbf{Comprehensive Evaluation}: We demonstrated 11,327$\times$ lower
    command latency and 2.7$\times$ higher mixed-workload throughput compared
    to traditional GPU programming patterns.
\end{enumerate}

\subsection{Key Findings}

Our evaluation reveals that the choice between persistent actors and traditional
kernels depends on workload characteristics:

\begin{itemize}
    \item \textbf{Interactive workloads}: Persistent actors dominate due to
    near-zero command latency (0.03$\mu$s vs 317$\mu$s)

    \item \textbf{Batch computation}: Traditional kernels achieve higher throughput
    for pure computation without host interaction

    \item \textbf{Mixed workloads}: Persistent actors enable real-time applications
    (60 FPS) that are infeasible with traditional launch-per-operation patterns
\end{itemize}

\subsection{Significance}

RingKernel bridges two successful but previously separate paradigms:

\begin{itemize}
    \item The \textbf{actor model}---proven for building reliable concurrent systems
    (Erlang, Akka, Orleans)

    \item \textbf{GPU computing}---proven for massive parallelism and throughput
    (CUDA, OpenCL)
\end{itemize}

By unifying these paradigms, RingKernel enables developers to apply familiar actor
patterns to GPU hardware, unlocking new applications that require both massive
parallelism and responsive interaction.

\subsection{Availability}

RingKernel is open-source software available at:

\begin{center}
\url{https://github.com/mivertowski/RustCompute}
\end{center}

The repository includes:
\begin{itemize}
    \item Full source code (Apache 2.0 / MIT dual license)
    \item 900+ tests across the workspace
    \item Example applications (FDTD simulation, transaction monitoring)
    \item Documentation and tutorials
\end{itemize}

Published crates are available on crates.io under the \texttt{ringkernel-*} namespace.

\subsection{Closing Remarks}

Fifty years after Hewitt's original actor model paper, actors remain a powerful
abstraction for concurrent computation. GPUs offer unprecedented parallel
processing capability, yet have lacked the programming model support that makes
actors compelling. RingKernel takes a step toward closing this gap, demonstrating
that actor semantics and GPU execution can be unified productively.

We hope RingKernel inspires further research into high-level programming models
for heterogeneous computing, making GPU capabilities accessible to a broader
audience of developers.
